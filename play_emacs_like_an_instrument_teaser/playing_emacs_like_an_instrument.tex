% Created 2019-10-27 Sun 08:26
% Intended LaTeX compiler: pdflatex
\documentclass[bigger]{beamer}
\usepackage[utf8]{inputenc}
\usepackage[T1]{fontenc}
\usepackage{graphicx}
\usepackage{grffile}
\usepackage{longtable}
\usepackage{wrapfig}
\usepackage{rotating}
\usepackage[normalem]{ulem}
\usepackage{amsmath}
\usepackage{textcomp}
\usepackage{amssymb}
\usepackage{capt-of}
\usepackage{hyperref}
\usepackage{minted}
\usetheme{metropolis}
\author{\includegraphics[height=0.8cm]{images/Logo_200ok.png} \newline 200ok GmbH}
\date{Alain M. Lafon, 2019-10-26 \newline alain@200ok.ch}
\title{Play Emacs like an instrument \newline - Teaser -}
\hypersetup{
 pdfauthor={\includegraphics[height=0.8cm]{images/Logo_200ok.png} \newline 200ok GmbH},
 pdftitle={Play Emacs like an instrument \newline - Teaser -},
 pdfkeywords={beamer org orgmode},
 pdfsubject={},
 pdfcreator={Emacs 27.0.50 (Org mode 9.1.9)},
 pdflang={English}}
\begin{document}

\maketitle
\begin{frame}{Outline}
\tableofcontents
\end{frame}

\addtobeamertemplate{frametitle}{}{%
\begin{tikzpicture}[remember picture,overlay]
\node[anchor=north east,yshift=2pt] at (current page.north east) {\includegraphics[height=0.8cm]{images/Logo_200ok_white.png}};
\end{tikzpicture}}

% Call \framedgraphic with either {filename.jpg} or [size]{filename.jpg}
\newcommand{\framedgraphic}[2][0.7] {
  \center\includegraphics[width=\textwidth,height=#1\textheight,keepaspectratio]{#2}
}

\section{Introduction}
\label{sec:orga6ffc52}

\begin{frame}[label={sec:org38857b8}]{The tiger bites the thrower,\newline the dog chases the mud-ball}
\begin{block}{Proposition}
\begin{itemize}
\item Most work on the computer is based on either text processing or text
consumption
\item Not having a general text editor at your disposal is like being a
carpenter and only having a hammer in the toolbox
\end{itemize}
\end{block}
\end{frame}


\begin{frame}[label={sec:org2a4583a}]{The tiger bites the thrower,\newline the dog chases the mud-ball}
\begin{block}{Emacs is a great general text processor}
\begin{itemize}
\item Emacs is a \alert{Lisp REPL}
\item It is \alert{self documenting}
\item It can be \alert{changed drastically} by the user while it is running

\item To me, it is the Swiss Army knife, not of text processors, but
ultimately of programs
\end{itemize}
\end{block}
\end{frame}


\begin{frame}[label={sec:orga1b1457}]{Original talk}
\begin{itemize}
\item This \alert{lightning talk} is a teaser of the original talk
\item It'll give you an impression on whether it's worth for you to watch
the original talk or not
\end{itemize}

\center\rule{0.5\paperwidth}{0.4pt}

\begin{itemize}
\item I'll cover \alert{a lot} of ground in the original talk
\begin{itemize}
\item That's because I spend most of my \emph{computer time} in Emacs
\item Also, there will be lots of background-info up-front
\item It is \alert{not} a lightning talk
\begin{itemize}
\item It's about 2h long $\backslash$(\^{}\_\^{})/
\end{itemize}
\end{itemize}
\end{itemize}
\end{frame}



\section{Impressions}
\label{sec:org7515602}
\begin{frame}[label={sec:orgac33c62}]{Impressions}
\begin{block}{The faces of Emacs}
\begin{itemize}
\item Some impressions of the talk
\item I call them "The faces of Emacs"
\end{itemize}
\end{block}
\end{frame}


\begin{frame}[label={sec:orgc2dd868}]{The faces of Emacs}
\begin{block}{Inception (obviously)}
\framedgraphic[0.6]{images/emacs_demo_inception.png} \footnote{org-mode and pdf-tools}
\end{block}
\end{frame}

\begin{frame}[label={sec:org80bc5d7}]{The faces of Emacs}
\begin{block}{Mail (Text and HTML)}
\framedgraphic[0.6]{images/emacs_demo_mu4e.png} \footnote{mu4e}
\end{block}
\end{frame}

\begin{frame}[label={sec:orgfa78672}]{The faces of Emacs}
\begin{block}{Git}
\framedgraphic[0.6]{images/emacs_demo_magit.png} \footnote{magit (Also: Note how Emacs is configured with literate programming)}
\end{block}
\end{frame}

\begin{frame}[label={sec:org6677ae6}]{The faces of Emacs}
\begin{block}{Organization}
\framedgraphic[0.6]{images/emacs_demo_org1.png} \footnote{Spreadsheets, project planning, time tracking, etc with org-mode}
\end{block}
\end{frame}

\begin{frame}[label={sec:orgbb8850f}]{The faces of Emacs}
\begin{block}{Browsing the web (distraction/tracking free)}
\framedgraphic[0.6]{images/emacs_demo_eww.png} \footnote{Browsing \href{https://200ok.ch}{200ok.ch} with eww}
\end{block}
\end{frame}

\begin{frame}[label={sec:org555f19c}]{The faces of Emacs}
\begin{block}{Someone might even use Emacs for coding}
\framedgraphic[0.6]{images/emacs_demo_cider.png} \footnote{Test Clojure code with cider}
\end{block}
\end{frame}

\begin{frame}[label={sec:org169af7d}]{The faces of Emacs}
\center\textbf{Infinite diversity in infinite combinations}

\center Ok then, but what \alert{is} Emacs?
\end{frame}

\section{Definition time}
\label{sec:org2ec616a}

\begin{frame}[fragile,label={sec:org3f2af96}]{What even is Emacs?}
 \begin{block}{C-h i (Emacs FAQ -> Status of Emacs)}
\framedgraphic{images/emacs_definition.png}
\end{block}
\end{frame}

\begin{frame}[label={sec:orge9e8b1e}]{So\ldots{} Emacs is software from 1976?!}
\begin{block}{Every other old software is \alert{young} compared to that}
\begin{itemize}
\item The first VIM release was 1991
\begin{itemize}
\item Says \href{https://en.wikipedia.org/wiki/Vim\_(text\_editor)}{Wikipedia}\footnote{\url{https://en.wikipedia.org/wiki/Vim\_(text\_editor)}}, I couldn't find it in the VIM in-editor
docs, on vim.org the original VIM 1.0 source of thje git repo of
current VIM (8.0).
\end{itemize}
\item Linux appeared in 1991, as well\footnote{\url{https://en.wikipedia.org/wiki/Linux}}
\item Only C is (slightly) older (1972) \footnote{\url{https://en.wikipedia.org/wiki/C\_(programming\_language)}}
\end{itemize}
\end{block}
\end{frame}

\begin{frame}[fragile,label={sec:org3c6973e}]{GNU}
 \begin{block}{Emacs is the mother of all \alert{Free Software}}
\begin{columns}
\begin{column}{0.50\columnwidth}
\begin{block}{M-x describe-gnu-project}
\begin{itemize}
\item GCC
\item GNU
\item Emacs
\end{itemize}
\end{block}
\end{column}

\begin{column}{0.50\columnwidth}
\begin{block}{⋆}
\begin{itemize}
\item GPL
\item FSF
\item GNU/Linux
\end{itemize}
\end{block}
\end{column}
\end{columns}
\end{block}

\begin{block}{GNU Emacs}
The Emacs tutorial (C-h t) dates the copyright to 1985.
\end{block}
\end{frame}

\begin{frame}[label={sec:org4bb5d64}]{GNU}
\begin{block}{What is GNU?\footnote{\url{https://www.gnu.org/}}}
\begin{quote}
GNU is an operating system that is free software—that is, it respects
users' freedom. The development of GNU made it possible to use a
computer without software that would trample your freedom.
\end{quote}
\end{block}
\end{frame}


\section{Play Emacs like an instrument}
\label{sec:orge33b69c}

\begin{frame}[label={sec:org7eb833e}]{Live Demo}
\begin{block}{The meat part}
\begin{itemize}
\item Switching gears now into live Emacs usage
\item I plan to announce all the new features that I'm touching
\item If I'm loosing you, forget to mention the name of a feature or you
have any other questions, don't hesitate to ask!
\end{itemize}
\end{block}
\end{frame}

\begin{frame}[label={sec:org328c90c}]{Live Demo}
\begin{block}{Live Coding - what can go wrong?}
\framedgraphic{images/lambda_workplace.jpg}
\end{block}
\end{frame}

\section{Closing words}
\label{sec:orgf4fb0fc}

\begin{frame}[label={sec:org93c234b}]{Original Talk}
If you liked this short teaser, please check out the original talk
here:
\url{https://200ok.ch/posts/2018-04-27\_Play\_Emacs\_like\_an\_Instrument.html}
\end{frame}

\begin{frame}[fragile,label={sec:orgfba8a7d}]{Further reading}
 If you want to continue on your own:

\begin{itemize}
\item Emacs Tutorial: C-h t || M-x help-with-tutorial
\item Emacs Manual: C-h r || M-x info-emacs-manual
\item All info manuals: C-h i || M-x info
\begin{itemize}
\item org-mode
\item magit
\item mu4e
\item Actually all info-manuals from your system and Emacs
\end{itemize}

\item \href{https://200ok.ch/category/emacs.html}{200ok.ch/category/emacs.html}
\item \href{https://200ok.ch/atom.xml}{200ok.ch/atom.xml}
\end{itemize}
\end{frame}

\begin{frame}[label={sec:org58b5be0}]{Further reading}
\begin{block}{Configuration}
\framedgraphic[0.6]{images/emacs_demo_literate.png} \footnote{Literate Configuration/Programming through org-mode}
\end{block}
\end{frame}

\begin{frame}[label={sec:org2de4fed}]{Talk tax}
\begin{block}{Give yourself a chance, use Emacs!}
\begin{itemize}
\item If you liked this talk, head over to
\url{https://github.com/munen/emacs.d/}, go through the README and enjoy
magical powers
\item Ah, yes - and put a star on the repo, would ya?(;
\end{itemize}
\end{block}

\begin{columns}
\begin{column}{0.45\columnwidth}
\begin{block}{⋆}
\includegraphics[height=0.35\textheight]{images/gnu_listen_half.jpg}
\end{block}
\end{column}


\begin{column}{0.45\columnwidth}
\begin{block}{⋆}
\includegraphics[height=0.35\textheight]{images/emacs_logo.png}
\end{block}
\end{column}
\end{columns}
\end{frame}
\end{document}
